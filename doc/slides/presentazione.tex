% !TEX encoding = UTF-8
% !TEX program = pdflatex
% !TEX root = presentazione.tex
% !TeX spellcheck = it_IT

%----------------------------------------------------------------------------------------
%	PACKAGES AND MISC SETTINGS
%----------------------------------------------------------------------------------------

\documentclass[12pt]{beamer}
%\documentclass[12pt,aspectratio=169]{beamer}
%\documentclass[12pt,aspectratio=43]{beamer}
\usepackage[italian]{babel}
\usepackage[utf8]{inputenc}
\usepackage{tabularx}
\usepackage{booktabs}
\usepackage{graphicx}
\usepackage{alltt}
\usepackage{tikz}
\usepackage{macros/appendixnumberbeamer}
\usepackage[pageofpages=di,% String used between the current page and the
                         % total page count.
          bullet=circle,% Use circles instead of squares for bullets.
          titleline=true,% Show a line below the frame title.
          alternativetitlepage=true,% Use the fancy title page.
          titlepagelogo=unipd-pollo,% Logo for the first page.
          watermark=logo-unipd-int% Watermark used in every page.
          ]{beamerthemeTorino}
\hypersetup{
	pdftitle={Propagazione di messaggi tra veicoli con modello urbano realistico},
	pdfauthor={Marco Romanelli},
	pdfstartview={Fit},
	pdfstartpage=1,
	bookmarks=true,
	bookmarkstype=toc,
	bookmarksopen=true,
	pdfhighlight=/I,
	pdfpagemode=FullScreen
}

\graphicspath{{resources/}{figures/}}

%----------------------------------------------------------------------------------------
%	TITLE PAGE
%----------------------------------------------------------------------------------------

\author[Romanelli,Marco]{Marco Romanelli\\[.5\baselineskip]
\scriptsize Relatore: Prof.\ Claudio Enrico Palazzin\\
\scriptsize Co-relatore: Dott.\ Armir Bujari}

\title{Propagazione di messaggi tra veicoli con modello urbano realistico}
% \course{Laurea Magistrale in Informatica}
\institute{Università di Padova}
\date{7 dicembre 2017}

\begin{document}

\section{Introduzione}\label{sec:intro}
\begin{frame}[t,plain]
	\titlepage
\end{frame}

	%----------------------------------------------------------------------------------------
	%	BODY
	%----------------------------------------------------------------------------------------

	\setcounter{framenumber}{0} % start from here
	\section{Contenuto}\label{sec:contenuto}
	% !TEX encoding = UTF-8
% !TEX program = pdflatex
% !TEX root = presentazione.tex
% !TeX spellcheck = it_IT
%
\begin{frame}[t]{Amazon Artificial Intelligence}
	intro
\end{frame}
%
%----------------------------------------------------------------------------------------
%	APPUNTI:
%		-
%----------------------------------------------------------------------------------------


	\section{Fine}\label{sec:fine}
	% !TEX encoding = UTF-8
% !TEX program = pdflatex
% !TEX root = presentazione.tex
% !TeX spellcheck = it_IT
%
\nonumber
\begin{frame}[t,plain,noframenumbering]{Grazie per l'attenzione}
\begin{figure}[f]
\centering
    \includegraphics[width=.5\textwidth,keepaspectratio=true]{unipd-pollo}
\end{figure}
\end{frame}
%
%----------------------------------------------------------------------------------------
%	APPUNTI:
% 		-
%----------------------------------------------------------------------------------------


	%----------------------------------------------------------------------------------------
	%	END
	%----------------------------------------------------------------------------------------

\end{document}
