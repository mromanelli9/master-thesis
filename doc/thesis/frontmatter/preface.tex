%!TEX TS-program = xelatex
%!TEX encoding = UTF-8 Unicode
%!TeX spellcheck = it_IT
%!TEX root = ../tesi.tex

% SOMMARIO
% I recenti progressi tecnologici nell'hardware, software e nell'ambito delle telecomunicazioni hanno permesso
% l'ideazione e lo studio di vari tipologie di reti, impiegate poi in un'ampia gamma di scenari.
Le reti veicolari (\textit{Vehicular Ad-hoc Networks}, VANET) stanno riscuotendo un crescente interesse
grazie ai recenti progressi tecnologici e ai molti campi d'applicazione possibili, dalla sicurezza dei veicoli
e delle strade allo sviluppo di servizi multimediali intra-veicolari.
% Vista l'impossibilità per un'infrastruttura centrale di garantire una copertura totale, un settore della ricerca % mah
% verte sullo studio di protocolli di comunicazione intelligenti in grado di propagare informazioni velocemente.
% Per la progettazione e la verifica di questi spesso si utilizzano simulazioni software in grado di analizzare
% diversi scenari limitando i costi di sviluppo.
Affinché tali reti rappresentino al meglio la complessità del mondo reale, si rende necessario lo sviluppo di modelli di propagazione del segnale
che considerino i vari effetti di degradazione, come la distanza, cammini multipli da riflessione, ombreggiatura.
Proprio quest'ultima ricopre un ruolo essenziale se lo scenario considerato è urbano o suburbano; in tal caso, infatti,
l'attenuazione del segnale radio causata dagli edifici non può essere ignorata.
Prendendo uno dei modelli a ostacoli presenti in letteratura,
combinandolo con dati reali sulla topologia stradale e sulla geometria degli edifici
si ottengono simulazioni più fedeli alla realtà, mantenendo tuttavia limitato il loro costo computazionale.
Preso in esame un protocollo per la propagazione di messaggi, chiamato Fast Broadcast,
si andrà a studiarne il comportamento in presenza del modello a ostacoli in diversi scenari urbani.
%
% TODO, NB: Sotto sono risultati, da controllare alla fine!
% I risultati mostrano come includendo questi fattori nella simulazione migliori la valutazione sulle prestazioni del protocollo in esame.
