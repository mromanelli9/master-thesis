%!TEX TS-program = xelatex
%!TEX encoding = UTF-8 Unicode
%!TeX spellcheck = it_IT
%!TEX root = ../tesi.tex

% SOMMARIO
% I recenti progressi tecnologici nell'hardware, software e nell'ambito delle telecomunicazioni hanno permesso
% l'ideazione e lo studio di vari tipologie di reti, impiegate poi in un'ampia gamma di scenari.
Le reti veicolari (\textit{Vehicular Ad-hoc Networks}, VANET) stanno, negli ultimi anni, riscuotendo un crescente interesse
grazie ai recenti progressi tecnologici e ai molteplici casi di studio, dalla sicurezza dei veicoli e delle strade e allo sviluppo di applicazioni
multimediali intra-veicolari.
% Vista l'impossibilità per un'infrastruttura centrale di garantire una copertura totale, un settore della ricerca % mah
% verte sullo studio di protocolli di comunicazione intelligenti in grado di propagare informazioni velocemente.
% Per la progettazione e la verifica di questi spesso si utilizzano simulazioni software in grado di analizzare
% diversi scenari limitando i costi di sviluppo.
Affinché tali reti rappresentino al meglio la complessità del mondo reale, si rende necessario lo sviluppo di modelli di propagazione del segnale
che considerino i vari effetti di degradazione, come la distanza, cammini multipli da riflessione, ombreggiatura.
Uno di questi, chiamato modello \textit{Obstacle Shadowing}, calcola l'attenuazione del segnale di propagazione dovuta alla presenza di ostacoli, come per esempio edifici.
Combinando quest'ultimo con un protocollo per la propagazione di messaggi e dati reali sulla geometria del'ambiente sottostante si ottiene una simulazione di uno scenario urbano più realistico.
%
% TODO, NB: Sotto sono risultati, da controllare alla fine!
I risultati mostrano come includendo questi fattori nella simulazione migliori la valutazione sulle prestazioni del protocollo in esame. 
