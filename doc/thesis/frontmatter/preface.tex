%!TEX TS-program = xelatex
%!TEX encoding = UTF-8 Unicode
%!TeX spellcheck = it_IT
%!TEX root = ../tesi.tex

% SOMMARIO
% I recenti progressi tecnologici nell'hardware, software e nell'ambito delle telecomunicazioni hanno permesso
% l'ideazione e lo studio di vari tipologie di reti, impiegate poi in un'ampia gamma di scenari.
Le reti veicolari (\textit{Vehicular Ad-hoc Networks}, VANET) stanno riscuotendo un crescente interesse
grazie ai recenti progressi tecnologici e ai molti campi d'applicazione possibili, dalla sicurezza dei veicoli
e delle strade allo sviluppo di servizi multimediali intra-veicolari.
All'intero di questo contesto è stato ideato Fast Broadcast,
un protocollo che sfrutta una stima del raggio trasmissivo per la rapida propagazione di informazioni.
Per rappresentare al meglio il mezzo fisico si rende necessario lo sviluppo di modelli di propagazione del segnale
che prendano in considerazione i vari effetti di degradazione, come la distanza, cammini multipli da riflessione, ombreggiatura da ostacoli.
Quest'ultima ricopre un ruolo essenziale se lo scenario considerato è urbano o suburbano,
dove gli edifici costitiscono un'ostruzione che non può essere ignorata.
Uno dei modelli per l'ombreggiatura da ostacoli presenti in letteratura
sfrutta dati reali sulla topologia stradale e sulla geometria degli ostacoli per ottenere simulazioni più realistiche.
Si andrà, quindi, a valutare il protocollo Fast Broadcast all'interno di due diversi scenari urbani reali.
Successivamente, sarà presa in considerazione la possibilità che alcuni veicoli non siano in grado
di partecipare allo scambio dei messaggi, includendo all'interno dello scenario
una rete di sensori per aiutare la propagazione.
Per permettere la possibilità che questi dispositivi siano posizionati a un'altezza diversa dal suolo
si andrà a proporre un'estensione al modello a ostacoli che consideri lo spazio tridimensionale.
