%!TEX TS-program = xelatex
%!TEX encoding = UTF-8 Unicode
%!TeX spellcheck = it_IT
%!TEX root = ../tesi.tex

\chapter{Simulazioni}\label{chap:simulazioni}
% INTRO: cosa c'è in questa sezione?
Dopo aver illustrato il protocollo in esame e il modello di propagazione utilizzato, si passa ora alla fase di valutazione.
Il capitolo procederà, quindi, nel dettaglio delle diverse simulazioni effettuate analizzando i relativi risultati.
% Ogni capitolo presenterà un gruppo di simulazioni, che differiscono dagli altri per scenario applicativo e configurazione.
% NB: non è bellissimo
%
\section{Configurazione a griglia}\label{sec:configurazione-griglia}
% \subsection{Panoramica}
% Simulazioni codice barichello:
%  - struttura simulazioni
%  - grafici
%  - risulati
Il primo gruppo di simulazioni ha lo scopo di analizzare il comportamento del protocollo Fast Broadcast con e senza edifici all'interno
di un ambiente conosciuto;
per questo motivo, la configurazione è la stessa presentata nel documento originale \cite{Barichello2017propagazione}, a cui sono stati aggiunti gli edifici.
Lo scenario è una città fittizia con strade a griglia in stile Manhattan di lunghezza $4000$ metri e distanti l'una dall'altra $300$ metri.
I veicoli sono disposti a $12$ metri di distanza per un totale di $8064$.
Seguendo l'idea originale, è stata definita anche un circonferenza di raggio pari a $1000\pm12$ metri, utilizzata per definire alcune metriche che verranno descritte in seguito.
Il raggio trasmissivo effettivo assume due valori possibili: $300$ metri o $500$ metri;
il tutto è riassunto in Tabella~\ref{tab:parametri-simulazioni-barichello}.

Per valutare il protocollo sono stati presi in esame diversi parametri: la copertura totale e la copertura sulla circonferenza,
il numero di salti necessari per raggiungere il bordo della griglia, il numero totale di messaggi di inoltro ricevuti e inviati.
In modo da aver un metro di paragone per il protocollo Fast Broadcast, quest'ultimo è stato comparato con altri due metodi, chiamati \statica e \staticb,
che utilizzano una stima fissa del raggio tramissivo, rispettivamente di $300$ e $500$ metri.

Gli edifici sono stati creati apposititamente, in modo che ogni edificio fosse contenuto all'interno dello spazio creato dalle strade.
Così facendo, risultano $169$ edifici di $295$x$295$ metri (i muri sono distanziati di $5$ metri dalla strada).
%
\begin{table}[htbp]
	\centering
	\begin{tabular}{| m{.4\linewidth} | p{.2\linewidth} |}
		\toprule
		Parametro											&			Valore [metri]			\\
		\thickerline
		Lunghezza delle strade				&			$4000$							\\
		Distanza fra le strade				&			$300$								\\
		Distanza fra i veicoloi 			&			$12$ 								\\
		Raggio trasmissivo effettivo	&			$300$ - $500$				\\
		Numero di simulazioni					&			$50$								\\
		\bottomrule
	\end{tabular}
	\caption{Configurazione dei parametri per le simulazioni.\label{tab:parametri-simulazioni-barichello}}
\end{table}
%
% \begin{figure}[htbp]
% 	\centering
% 	\begin{center}
% 		\includegraphics[width=.4\textwidth]{griglia-dettaglio-2.png}
% 	\end{center}
% 	\label{fig:griglia-dettaglio}\caption{Dettaglio della configurazione a griglia: in nero le strade e in giallo gli edifici.}
% \end{figure}
%
\subsection{Risultati}\label{sec:configurazione-griglia-risultati}
%
% TODO
%
\section{Scenario urbano reale} % un titolo migliore no?
Le configurazioni precedenti avevano il difetto di essere poco veritiere, sia dal punto di vista della topologia stradale
che sulla geometria degli edifici.
In questo gruppo di simulazioni, invece, si è voluto valutare Fast Broadcast all'interno di uno scenario quanto più reale possibile.
Sono state scelte due città, Padova (IT) e Los Angeles (California, USA), definite due aree nella zona centrale di circa $5$ kilometri quadrati (Figura~\ref{fig:scenari-la-pd-osm})
e, seguendo il procedimento descritto nella Sezione~\ref{sec:sumo}, si sono estratte le informazioni necessarie alla simulazione.
I parametri risultanti sono elencati in Tabella~\ref{tab:parametri-simulazioni-pd-la}.

La scelta è ricaduta su Los Angeles poiché ha una rete stradale a griglia in stile Manhattan, simile anche allo scenario precedente,
mentre Padova (sede anche dell'Università dove si è svolto questo lavoro) ha una toplogià più irregolare, con strade più strette ed edifici a ridosso di queste,
zone pedonali e ZTL.
%
\begin{table}[htbp]
	\centering
	  \begin{tabular}{| m{.35\linewidth} | p{.01\linewidth} | p{.04\linewidth} | p{.15\linewidth} | p{.15\linewidth} |}
			\toprule
			\multicolumn{3}{|m{.3\linewidth}|}{\multirow{2}{*}{}}														&		\multicolumn{2}{c|}{Scenario}						\\ \cline{4-5}
			\multicolumn{3}{|m{.3\linewidth}|}{}																						&		Padova				&			Los Angeles					\\
			\thickerline
			\multicolumn{2}{|m{.25\linewidth}|}{\multirow{2}{*}{Latitudine}}				&		N	 	& 	$45,4171$				&			$33,9654$					\\ \cline{3-5}
			\multicolumn{2}{|m{.25\linewidth}|}{}																		&		S	 	& 	$45,3981$				&			$33,9478$					\\ \hline
			\multicolumn{2}{|l|}{\multirow{2}{*}{Longitudine}}											&		O	 	& 	$11,8654$				&			-$118,3260$				\\ \cline{3-5}
			\multicolumn{2}{|l|}{}																									&		E	 	& 	$11,8923$				&			-$118,3055$				\\ \hline
			\multicolumn{3}{|l|}{Area approssimativa (km$^2$)}															&		\multicolumn{2}{c|}{$5$}								\\ \hline
			\multicolumn{3}{|l|}{Distanza fra veicoli (metri)}															&		\multicolumn{2}{c|}{$25$}								\\ \hline
			\multicolumn{3}{|l|}{Numero di veicoli}																					&		$2224$					&					$1905$				\\ \hline
			\multicolumn{3}{|l|}{Numero di edifici}																					&		$6322$					&					$8241$				\\ \hline
			\multicolumn{3}{|l|}{Numero di simulazioni}																			&		\multicolumn{2}{c|}{$50$}								\\
			\bottomrule
	  \end{tabular}
	\caption{Parametri per gli scenari urbani.\label{tab:parametri-simulazioni-pd-la}}
\end{table}
%
\begin{figure}[htbp]
	\centering
		\includegraphics[width=.49\textwidth]{osm_web-pd-2x2.png}
		\hfill
		\includegraphics[width=.49\textwidth]{osm_web-la-2x2.png}
\caption{Una vista delle aree selezionate per le simulazioni; a sinistra la città di Padova e a destra Los Angeles (fonte: OSM).\label{fig:scenari-la-pd-osm}}
\end{figure}
