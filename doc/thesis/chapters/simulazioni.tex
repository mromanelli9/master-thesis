%!TEX TS-program = xelatex
%!TEX encoding = UTF-8 Unicode
%!TeX spellcheck = it_IT
%!TEX root = ../tesi.tex

\chapter{Simulazioni}\label{chap:simulazioni}
% INTRO: cosa c'è in questa sezione?
Dopo aver illustrato il protocollo in esame e il modello di propagazione utilizzato, si passa ora alla fase di valutazione.
Il capitolo procederà, quindi, nel dettaglio dei diversi scenari affrontati illustrando
le motivazioni che hanno portato alla loro realizzazione, definendone i parametri e motivando i risulati ottenuti.
%
\section{Scenario a griglia}\label{sec:configurazione-griglia}
% \subsection{Panoramica}
% Simulazioni codice barichello:
%  - struttura simulazioni
%  - grafici
%  - risulati
Il primo gruppo di simulazioni ha lo scopo di analizzare l'impatto degli edifici sul comportamento del protocollo
Fast Broadcast all'interno di uno scenario conosciuto, ossia quello presentato nella tesi originale~\cite{Barichello2017propagazione}
a cui sono stati aggiunti degli edifici creati apposititamente.
La configurazione dello scenario, dei nodi e della rete è la stessa ed è riassunta in Tabella~\ref{tab:parametri-simulazioni-barichello}.
L'ambiente è una città fittizia con strade a griglia in stile Manhattan di lunghezza $4000$ metri e distanti l'una dall'altra $300$ metri.
I veicoli sono disposti a $12$ metri di distanza per un totale di $8064$.
Il veicolo che da inizio alla fase di inoltro (generazione del primo messaggio di inoltro),
che per facilità verrà chiamato veicolo \textit{zero}, è posizionato al centro della griglia.
Seguendo l'idea originale, è stata definita anche un circonferenza di raggio pari a $1000\pm12$ metri, utilizzata per definire alcune metriche che verranno descritte in seguito;
la circonferenza ha centro in corrispondenza del veicolo zero.
Il raggio trasmissivo effettivo (fisico) assume due valori possibili: $300$ o $500$ metri;
%
\begin{table}[!h]
	\centering
	\begin{tabular}{| L{.4\linewidth} | r  l |}
		\toprule
		Parametro															&			Valore 							&					\\
		\thickerline
		Lunghezza delle strade								&			$4000$							& m				\\
		Distanza fra le strade								&			$300$								& m				\\
		Distanza fra i veicoloi								&			$12$ 								& m				\\
		Circonferenza													&			$1000\pm12$					& m				\\
		Posizione del veicolo zero						&			centrale						&					\\
		\thickerline
		Dimensioni pacchetto									&				$164$							&			byte		\\	\hline
		Standard tramissione									&				$802.11$b					&							\\	\hline
		Frequenza															&				$2.4$							&			GHz			\\	\hline
		Banda del canale											&				$22$							&			MHz			\\	\hline
		Velocità di tramissione								&				$11$							&			Mbps		\\	\hline
		Potenza trasmissione									&				$7,5$							&			dBm			\\	\hline
		Raggio trasmissivo										&				$300$-$500$				&			m				\\	\hline
		Codifica															&				DSSS\footnotemark	&							\\	\hline
		Modello di propagazione								&				\textsf{ns3::RangePropagation}	&							\\	\hline
		Modello di ombreggiatura							&				A ostacoli				&							\\	\hline
		\thickerline
		Simulazioni	per configurazione				&			$50$								&					\\
		\bottomrule
	\end{tabular}
	\caption{Configurazione dei parametri per le simulazioni.\label{tab:parametri-simulazioni-barichello}}
\end{table}
\footnotetext{\textit{Direct Sequence Spread Spectrum}}	% TODO controllare la posizone, che sia nella stessa pagina della tabella.
%
Per valutare il protocollo sono stati presi in esame diversi parametri: la copertura totale e la copertura sulla circonferenza,
il numero di salti necessari per raggiungere il bordo della griglia, il numero totale di messaggi di inoltro ricevuti e inviati.
In modo da aver un metro di paragone per il protocollo Fast Broadcast, quest'ultimo è stato comparato con altri due metodi, chiamati \statica e \staticb,
che utilizzano una stima fissa del raggio tramissivo rispettivamente di $300$ e $500$ metri.
Il raggio trasmissivo viene variato agendo sul modello di propagazione \textsf{ns3::RangePropagation},
nel quale una trasmissione viene ricevuta se è a una distanza minore o uguale della portata impostata.
Gli edifici, non presenti nello scenario originale, sono stati creati manualmente, in modo che ogni edificio fosse contenuto all'interno dello spazio creato dalle strade.
Così facendo, risultano $169$ edifici di $295$x$295$ metri (i muri sono distanziati di $5$ metri dalla strada).

Le simulazioni sono state eseguite sfruttando l'attrezzatura in dotazione al servizio di \textit{High Performance Computing} dell'università di Padova,
% mettere anche il tempo?
eseguendono un numero pari a $50$ simulazioni per ogni configurazione.
%
% \begin{figure}[htbp]
% 	\centering
% 	\begin{center}
% 		\includegraphics[width=.4\textwidth]{griglia-dettaglio-2.png}
% 	\end{center}
% 	\label{fig:griglia-dettaglio}\caption{Dettaglio della configurazione a griglia: in nero le strade e in giallo gli edifici.}
% \end{figure}
%
\subsection{Risultati}\label{sec:configurazione-griglia-risultati}
% I grafici in \figurename~\ref{fig:risultati-griglia-copertura} mostrano il cambiamento della copertura dei veicoli in totale e sulla circonferenza,
% dove si hanno rispettivamente $8064$ e $36$ veicoli, quando si vanno a considerare gli edifici.
% Nelle simulazioni rappresentato a sinistra
% Si può notare un generale calo della copertura causato dalla perdita di segnale per effetto dell'ombreggiatura,
% anche se non significativo; la differenza è leggermente maggiore considerando solo i veicoli sulla circonferenza.
% Analizzando invece il numero dei salti (\figurename~\ref{fig:risultati-griglia-salti}) necessari a raggiungere il perimetro della griglia
% si nota un aumento generale di questo valore, leggermente maggiore con raggio trasmissivo $300$ metri.
% Questo comporta che, mediamente, ci vogliono $1,104$ salti in più per raggiungere il perimetro se il raggio
% trasmissivo è di $300$m, mentre ce ne vogliono $0,571$ se è di $500$m.
% Il grafico in \figurename~\ref{fig:risultati-griglia-messaggi}) mostra un aumento del numero di messaggi
% di inoltro inviati, in media, del $17,9$\% nel caso del raggio trasmissivo di $300$m
% e del $163,9$\% nel secondo caso.
% Combinando questi risultati con i precenti, si denota che, nel caso degli edifici,
% un alto numero di pacchetti venga perso a causa della troppa attenuazione del segnale;
% infatti per coprire (circa) la stessa percentuale di veicoli è necessario un numero molto maggiore di inoltri.
I grafici dalla \figurename~\ref{fig:risultati-griglia-copertura} alla~\ref{fig:risultati-griglia-messaggi}
mostrano i risultati di questo gruppo di simulazioni.
Come si può notare, la presenza degli ostacoli influisce molto poco sui quasi tutti i parametri presi in esame.
La differenza sulla copertura totale è, in generale, inferiore o vicina all'$1\%$, tuttavia utilizando i protocolli FB e \statica{} si vede
un leggero calo nel caso di raggio tramissivo pari a $300$m e un leggero incremento con $500$m, mentre
è l'opposto con \staticb{}; la differenza è comunque minima per ipotizzare una causa di questo comportamento. % ?
I veicoli sulla circonferenza (di raggio $1000$m) ottengono una copertura legggermente migliore
(in media dello $0,5\%$), tranne nel caso di \staticb{} con raggio trasmissivo $500$m.
Questo leggero aumento della copertura potrebbe essere dovuto a quello che gli autori del modello a ostacoli (~\cite{Carpenter:2015:OMI:2756509.2756512})
chiamano ``L'effetto cena con amici'' (\textit{The Dinner Party effect}): la presenza di ostacoli come edifici
può aumentare, localmente, la probabilità di tramissioni con successo, in quanto aumenterebbe il riuso delle spazio.
Tale fenomeno può anche essere visto nel grafico in \figurename~\ref{fig:risultati-griglia-messaggi}
dove nel caso di raggio trasmissivo pari a $500$m è presente un aumento, in media, del $5,6$\%
nel numero di messaggi di intoltro ricevuti, mentre quelli inviati sono più simili al caso senza edifici.
Anche il numero di salti necessari a raggiungere il perimetro della griglia rimane vicino
e aumenta mediamente solo del $0,34\%$.
Da tutto ciò si deduce che la presenza di edifici causa un aumento locale del traffico di messaggi
ma il meccanimo di contesa e inoltro permette di mantenere limitata la propagazione di questi in avanti. % sicuro?
Ciascun edificio ricopre un intero isolato (spazio interno racchiuso fra le strade) e si trova, per una differenza di pochi metri,
a ridosso delle strade;
la grandezza di questi edifici probabilmente blocca qualsiasi trasmissione fra due veicoli che si trovano
agli estremi dell'isolato e, di conseguenza, limita la propagazione del messaggio all'interno del blocco  che viene ``forzato''
a proseguire lungo la strada.
%
\begin{figure}[htbp]
	\centering
		\includegraphics[width=\linewidth]{grafici/griglia/copertura_totale.png}
		\includegraphics[width=\linewidth]{grafici/griglia/copertura_circonferenza.png}
\caption{Scenario a griglia: copertura dei veicoli in totale e sulla circonferenza.\label{fig:risultati-griglia-copertura}}
\end{figure}
%
\begin{figure}[htbp]
	\centering
		\includegraphics[width=\linewidth]{grafici/griglia/salti.png}
\caption{Scenario a griglia: numero di salti.\label{fig:risultati-griglia-salti}}
\end{figure}
%
\begin{figure}[htbp]
	\centering
		\includegraphics[width=\linewidth]{grafici/griglia/messaggi_inviati.png}
		\includegraphics[width=\linewidth]{grafici/griglia/messaggi_ricevuti.png}
\caption{Scenario a griglia: numero di messaggi di inoltro inviati e ricevuti.\label{fig:risultati-griglia-messaggi}}
\end{figure}
\clearpage
%
%
\section{Scenario urbano reale} % un titolo migliore no?
Le configurazioni precedenti avevano il difetto di essere poco veritiere, sia dal punto di vista della topologia stradale
che sulla geometria degli edifici.
Le strade formavano una griglia perfetta, con strade identiche e incroci esatti;
gli edifici erano molto grandi e tutti della stessa forma e dimensione.
In questo gruppo di simulazioni, invece, si è cercato di rappresentare un vero scenario realistico,
con strade di lunghezza diversa e non perfettamente allineate,
con edifici meno squadrati, a ridosso della strada o anche assenti in alcune zone (difficile in città, ma possibile).

A questo proposito sono state scelte due città, Padova (IT) e Los Angeles (California, USA) (Figura~\ref{fig:scenari-la-pd-osm}), definite due aree nella zona centrale di circa $5$ kilometri quadrati (Figura~\ref{fig:scenari-la-pd-osm})
e, seguendo il procedimento descritto nella Sezione~\ref{sec:sumo}, sono state estratte le informazioni necessarie alla configurazione delle simulazioni.
Altre informazioni sugli scenari e sui parametri delle rete sono elencati in Tabella~\ref{tab:parametri-simulazioni-pd-la}.
La scelta è ricaduta su Los Angeles poiché la sua rete stradale è paragonabile a una griglia in stile Manhattan, come nello scenario precedente,
mentre Padova (sede anche dell'Università dove si è svolto questo lavoro) ha una toplogià più irregolare, con strade più strette ed edifici a ridosso di queste,
zone pedonali e ZTL.
In questo modo si ha il primo scenario che si discosta meno dal precendete, mentre il secondo in modo più deciso.
%
\begin{table}[htbp]
	\centering
	  \begin{tabular}{| L{.35\linewidth} | C{.01\linewidth} | C{.04\linewidth} | L{.15\linewidth} | L{.15\linewidth} |}
			\toprule
			\multicolumn{3}{|m{.3\linewidth}|}{\multirow{2}{*}{}}														&		\multicolumn{2}{c|}{Scenario}						\\ \cline{4-5}
			\multicolumn{3}{|m{.3\linewidth}|}{}																						&		Padova				&			Los Angeles					\\
			\thickerline
			\multicolumn{2}{|m{.25\linewidth}|}{\multirow{2}{*}{Latitudine}}				&		N	 	& 	$45,4171$				&			$33,9654$					\\ \cline{3-5}
			\multicolumn{2}{|m{.25\linewidth}|}{}																		&		S	 	& 	$45,3981$				&			$33,9478$					\\ \hline
			\multicolumn{2}{|l|}{\multirow{2}{*}{Longitudine}}											&		O	 	& 	$11,8654$				&			-$118,3260$				\\ \cline{3-5}
			\multicolumn{2}{|l|}{}																									&		E	 	& 	$11,8923$				&			-$118,3055$				\\ \hline
			\multicolumn{3}{|l|}{Area approssimativa [km$^2$]}															&		\multicolumn{2}{c|}{$5$}								\\ \hline
			\multicolumn{3}{|l|}{Distanza fra veicoli [metri]}															&		\multicolumn{2}{c|}{$25$}								\\ \hline
			\multicolumn{3}{|l|}{Numero di veicoli}																					&		$2224$					&					$1905$				\\ \hline
			\multicolumn{3}{|l|}{Numero di edifici}																					&		$6322$					&					$8241$				\\ \hline
			\multicolumn{3}{|l|}{Potenza trasmissione [dB]}																	&		\multicolumn{2}{c|}{$4,6$-$13,4$}						\\	\hline
			\multicolumn{3}{|l|}{Modello di propagazione}																		&		\multicolumn{2}{c|}{\textsf{ns3::TwoRayGround}}		\\	\hline
			\thickerline
			\multicolumn{3}{|l|}{Simulazioni	per configurazione}														&		\multicolumn{2}{c|}{$50$}					\\
			\bottomrule
	  \end{tabular}
	\caption{Parametri della topologia per gli scenari urbani.\label{tab:parametri-simulazioni-pd-la}}
\end{table}
%
\begin{figure}[htbp]
	\centering
		\includegraphics[width=.49\textwidth]{osm_web-pd-2x2.png}
		\hfill
		\includegraphics[width=.49\textwidth]{osm_web-la-2x2.png}
\caption{Una vista delle aree selezionate per le simulazioni; a sinistra la città di Padova e a destra Los Angeles (fonte: OSM).
Il cerchio rosso evidenzia la posizione del nodo di partenza mentre i tre rimanenti rappresentano il perimetro della circonferenza
e i due intervalli di confidenza.\label{fig:scenari-la-pd-osm}}
\end{figure}
%
Anche in questo gruppo di simulazioni sono state utilizzate le medesime metriche del caso precendente,
fatta eccezione per il numero di salti che ora sono per raggiungere la circonferenza, non il bordo dello scenario.
Questo perché nella configurazione a griglia lungo il perimetro esterno erano sicuramente posizionati dei veicoli,
mentre ora il limite ``quadratico'' dell'area è ideale (non tutte le strade finisco esattamente sul bordo della mappa).
%
\subsection{Los Angeles}\label{subsec:risultati-la}
A differenza del caso precente, quando si va a considerare l'effetto dell'ombreggiatura in questo scenario
si notano dei maggiori cambiamenti.
In primo luogo, come si può vedere dalla \figurename~\ref{fig:risultati-la-copertura},
si nota una diminuzione della copertura dei veicoli, più marcata con raggio trasmissivo $300$m.
Nelle simulazioni, infatti, le due diverse distanze massime raggiungibili dal segnale,
$300$ e $500$ metri senza ostruzioni sono state implementate tramite una diversa potenza in fase di trasmissione.
Un sengale più potente è, naturalmente, in grado di ``superare'' un edificio rispetto a un sengale con meno potenza.
Inoltre, avendo una portata maggiore potrebbe raggiungere altri elementi in ricezione e coprire quindi ulteriori veicoli e/o aree.

Da notare è il cambio di tendenza nel caso del protocollo \statica{}:
in presenza di edifici si ha una copertura maggiore con raggio trasmissivo di $500$m.
Il motivo potrebbe essere che l'effetto dell'ombreggiatura influisce maggiormente rispetto
allo non stimare correttamente il mezzo fisico.

Per quanto riguarda il numero medio di salti necessari a raggiungere i veicoli sulla circonferenza,
\figurename~\ref{fig:risultati-la-salti},
% TODO: contorllare che sia così anceh avendo nuovi dati
si ha un incremento del $36,5\%$ nel caso di raggio trasmissivo $300$m mentre solo del $5\%$ nella controparte.
Inoltre si può osservare come il protocollo FB abbia prestazioni migiori delle alternative statiche,
di circa il $10\%$ con $300$m e del $23\%$ con $500$m. % ha senso confrontarli così?

Considerando il numero di messaggi di inoltro che vengono scambiati rete (\figurename~\ref{fig:risultati-la-messaggi}),
rimane anomalo il caso \statica{}-$500$m, dove si ha un valore inferiore rispetto agli altri protocolli,
anche se rimane coerente nel considerare l'effetto ombreggiatura (rispetto allo scenario senza).
Mediamente, i messaggi inviati nel caso $300$m subiscono un incremento del $194,5\%$ e del $179,4\%$
con $500$m.
Un andamento inverso caratterizza, invece, il numero totale di messaggi ricevuti:
se si considera un raggio trasmissivo di $500$m si ha un aumento del $43\%$ in presenza di edifici,
mentre nella controparte a $300$m un calo del $32,2\%$.
Ciò significa che, nel primo caso, gli ostacoli considerati causano la perdita di una grossa quantità di messaggi
mentre così non avviene nel secondo.
La causa di questo fenomeno potrebbe trovarsi nella topologia stradale presente in questo specifico scenario.
Gli isolati più piccoli hanno una dimensioni di circa $50$x$175$ metri mentre quelli più grandi
(visibili nella mappa in basso a destra) sono lunghi circa $400$ metri.
Il raggio trasmissivo minore non riesce a superarli mentre è così per quello maggiore che raggiunge
di conseguenza anche i veicoli posizionati sugli e nelle vicinanze degli incroci.
Considerando gli isolati più piccoli, con $300$m si copre (al massimo) due incroci mentre se ne coprono
fino a tre se si hanno $500$m.
Inoltre, potrebbe essere che il segnale generato nel caso di raggio trasmissivo minore
non sia sufficiente a penetrare gli edifici presenti all'interno di un isolato (in larghezza),
mentre è abbastanza forte quello con raggio trasmissivo maggiore.
% i grafici su potenza-sengale-isolati-la lo confermeranno?

Anche in questo scenario, il protocollo FB si comporta secondo le aspettative
garantendo una copertura migliore % TODO da confermare dopo con i nuovi dati
rispetto ai protocolli con stima fissa,
un numero di salti inferiore e quindi una consegna del messaggio più veloce
e in generale un numero inferiore di messaggi inoltrati.
%
\begin{figure}[htbp]
	\centering
		\includegraphics[width=\linewidth]{grafici/la/copertura_totale.png}
		% \includegraphics[width=\linewidth]{grafici/la/copertura_circonferenza.png} % TODO da mettere!!!!
\caption{Scenario L.A.: copertura dei veicoli in totale e sulla circonferenza.\label{fig:risultati-la-copertura}}
\end{figure}
%
\begin{figure}[htbp]
	\centering
		\includegraphics[width=\linewidth]{grafici/la/salti.png}
\caption{Scenario L.A.: numero di salti.\label{fig:risultati-la-salti}}
\end{figure}
%
\begin{figure}[htbp]
	\centering
		\includegraphics[width=\linewidth]{grafici/la/messaggi_inviati.png}
		\includegraphics[width=\linewidth]{grafici/la/messaggi_ricevuti.png}
\caption{Scenario L.A.: numero di messaggi di inoltro inviati e ricevuti.\label{fig:risultati-la-messaggi}}
\end{figure}
\clearpage
%
%
\subsection{Padova}\label{subsec:risultati-pd}

% \begin{figure}[htbp]
% 	\centering
% 		\includegraphics[width=\textwidth]{grafici/pd_copertura_totale.png}
% 		\includegraphics[width=\textwidth]{grafici/pd_copertura_circonferenza.png}
% \caption{Copertura dei veicoli totale e sulla circonferenza dei veicoli\label{fig:risultati-padova-copertura}}
% \end{figure}
% %
% \begin{figure}[htbp]
% 	\centering
% 		\includegraphics[width=\textwidth]{grafici/pd_salti.png}
% \caption{Numero di salti necessario per raggiungere la circonferenza dei veicoli\label{fig:risultati-padova-salti}}
% \end{figure}
% %
% \begin{figure}[htbp]
% 	\centering
% 		\includegraphics[width=\textwidth]{grafici/pd_messaggi_inviati.png}
% 		\includegraphics[width=\textwidth]{grafici/pd_messaggi_ricevuti.png}
% \caption{Quantità di messaggi di inoltro durante la simulazione.\label{fig:risultati-padova-messaggi}}
% \end{figure}
