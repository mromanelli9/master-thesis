%!TEX TS-program = xelatex
%!TEX encoding = UTF-8 Unicode
%!TeX spellcheck = it_IT
%!TEX root = ../tesi.tex

\chapter{Introduzione}\label{chap:introduction}
% \section{Contesto}\label{sec:contesto}
% scaletta:
% - introdurre le vanet
% - importanza della propagazione/protocolli broadcast
% - importanza delle simulazioni
% - importanza di simulazioni veritiere
% - (~) modelli per rendere migliori le simulazioni
% - importanza degli ostacoli
% - modellazione di ambiente 3d
% - struttura documenti tesi
I progressi tecnologici dell'ultimo decennio nell'hardware, nel software e nelle telecomunicazioni hanno permesso
la larga diffusione di unità computazionali all'interno degli autoveicoli.% , anche in quelli di fascia medio-bassa.
Questo ha portato a un incremento dell'interesse della ricerca scientifica in questo campo e, in particolare,
sullo studio delle reti veicolari o VANET (\textit{Vehicular Ad-hoc Networks}).

Le VANET rappresentano una famiglia delle reti Ad-hoc mobili (\textit{Mobile Ad-hoc Network})
nelle quali i movimenti sono strutturati, i veicoli sono consapevoli della propria posizione nello spazio (ad esempio tramite geolocalizzazione GPS)
e sono equipaggiati con attrazzatura che permette le comunicazioni inter-veicolari (es.~antenne radio).
Queste reti sono alla base di una vasta gamma di necessità, che spaziano dalla sicurezza stradale alle applicazioni multimediali,
dalla distribuzione dei dati sul traffico allo sviluppo di una rete infrastrutturale urbana.

Una caratteristicha comunue è la necessità di un sistema di propagazione che trasporti informazioni
con il minor ritardo possibile; per esempio un pericolo inatteso sulla careggiata potrebbe causare l'impossibilità di proseguire
lungo quel percorso, oppure una trasmissione di contenuti multimediali (giochi mutiplayer, video) fra due o più veicoli~\cite{1580935}~\cite{PantelW02}.
Una delle soluzioni proposte si chiama Fast Broadcast: un protocollo per ridurre il tempo di propagazione
di un messaggio da una sorgente a una destinazione tramite una stima del raggio trasmissivo effettivo e
il conseguente utilizzo di questa per ridurre il numero di salti necessari al raggiungimento dell'obiettivo~\cite{Palazzi07howdo}.

La valutazione di protocolli di questo tipo su scenari urbani con migliaia di agenti risulta
problematica principalmente a causa dei costi proibitivi (oltre alle implicazioni sulla privacy)
che comporterebbe una sperimentazione nel mondo reale.
Spesso, quindi, i ricercatori ricadono sull'utilizzo di modelli simulati, eventualmente integrati con risultati ottenuti
da esperimenti sul campo in ambienti ridotti.
In particolare, l'accuratezza dei modelli di propagazione e di mobilità rappresenta la chiave per una buona valutazione delle prestazioni
dei protocolli di rete veicolari~\cite{4020783}.
All'interno di scenari urbani e suburbani, gli edifici ostruiscono la naturale propagazione di un segnale radio nello spazio
e, di conseguenza, al fin di eseguire simulazioni più affidabili questa ostruzione non può essere ignorata.
Una fra le diverse soluzioni proposte negli anni utilizza un semplice modello matematico
per il calcolo dell'attenuazione del segnale per un singolo edificio, lo combina con informazioni reali
sulla geometria degli edifici~\cite{Carpenter:2015:OMI:2756509.2756512}.
% NB,nota: è da dire che è su ns-3? sennò dopo come lo giustifico?

\paragraph{}
Il presente lavoro è strutturato come segue.
Nel capitolo successivo (Capitolo~\ref{chap:protocollo-fast-broadcast}) verrà presentato il protocollo Fast Broadcast e una proposta di estensione in due dimensioni.
Il Capitolo~\ref{chap:modello-a-ostacoli} analizzerà il modello a ostacoli, sia dal punto di vista matematico che da quello analitico inerente alla simulazione;
concluderà presentanto una possibile soluzione per rendere il modello tridimensionale.
Il Capitolo~\ref{chap:applicativi} darà una veloce panoramica sui software utilizzati e il metodo utilizzato per la creazione degli scenari.
Il Capitolo~\ref{chap:simulazioni} illusterà i diversi gruppi di simulazioni effettuati e i relativi risultati ottenuti.
Infine, nel Capitolo~\ref{chap:conclusioni} saranno presentate le conclusioni, evidenziando anche alcuni possibili sviluppi di questo lavoro.
%
\section{Modelli di radiopropagazione}\label{sec:modelli-propagazione}
Un modello di radiopropagazione (MRP) simula gli effetti dell'attenuazione del segnale radio (segnali elettromagnetici nello spazio libero o etere, in contrasto con la propagazione guidata)
dovuta alla distanza, cammini multipli per effetto della riflessione, ombreggiatura causata da grandi ostacoli come edifici.
L'utilizzo di un'idonea rappresentazione per questo tipo di ostacoli è, quindi, necessaria nel contesto di simulazioni di reti VANET (\textit{Vehicular Ad-hoc Networks}) in ambienti urbani
e suburbani.

Nel corso degli anni, diversi MRP sono stati proposti.
Il più semplice di questi si chiama modello a disco unitario (\textit{unit-disk model}), nel quale i veicoli possono comunicare fra loro se si trovano entro una certa soglia
di distanza, mentre non possono altrimenti~\cite{6554832}.
Un modello molto utilizzato nelle simuazioni di VANET è il modello a doppio raggio (\textit{Two-ray ground-reflection model}),
nel quale il segnale in fase di ricezione è composto da una componente in linea diretta e una seconda derivante dalla riflessione causata dal terreno~\cite{DBLP:books/daglib/0091821}.
In~\cite{Schmitz:2006:ERW:1164717.1164730} e in~\cite{Souley2005RealisticUS}, modelli più sviluppati prendono in considerazione anche le properità riflettenti delle superfici e degli ostacoli.

Tuttavia, un approcio diretto di questo tipo difficilmente scala al numero di nodi necessario in un classico scenario VANET e per questo motivo
spesso questi modelli si affidano a una fase di pre-elaborazione che può richiedere tempi elevati~\cite{Stepanov:2008:IMR:1293378.1293656}.
Per ovviare a questo problema, alcune proposte astraggono dalle informazioni sui singoli edifici, modellando l'ambiente urbano in modo omogeneo
così da creare un modello analitico per l'ombreggiatura~\cite{1492678}.
Questo tipo di modelli riducono il costo computazionale e generalmente si raffrontano bene con i risultati reali;
ciò nonostante, non riescono a catturare effetti a livello mesoscopico, come eventuali spazi fra edifici che permetterebbero trasmissioni a corto raggio~\cite{Giordano:2010:CST:1860058.1860065}.

Sono affetti dallo stesso problema anche i modelli (puramente) probabilistici, in quanto non tengono in considerazione la geometria urbana sottostante.
Questi modellano l'effetto dell'ombraggiatura tramite distribuzione stocastiche, fra cui Rice, Rayleigh, Nakagami-m, lognormale e Weibull~\cite{6554832}~\cite{Rappaport:2001:WCP:559977}.
%
\section{Modellazione di ostacoli}\label{sec:modellazione-ostacoli}
Nelle simulazioni di alcuni scenari, come possono essere le VANET, un'accurata rappresentazione della topologia dell'ambiente è necessaria,
in quanto questa limita non solo la mobilità dei veicoli ma interferisce anche con le trasmissioni radio~\cite{7543980}~\cite{amjad2015impact}.
L'attenuazione radio viene spesso modellata in modo deterministico basandosi principalmente sulla distanza della visuale (\textit{Line Of Sight}, LOS)
fra i veicoli, aggiungendo i modelli stocastici per considerare l'ombreggiatura.

Negli ultimi anni, diversi studi hanno cercato di rappresentare l'effetto ombreggiatura causato dagli edifici.
In~\cite{Giordano:2010:CST:1860058.1860065}, gli autori utilizzavano tecniche di geocodifica inversa (\textit{reverse geocoding}) per estrapolare
informazioni sulla topologia della rete stradale in stile Manhattan. Da qui, veniva derivata la geometria degli edifici utilizzata
per differenziare i casi in cui i due veicoli comunicanti avessero la visuale libera (LOS) o meno (\textit{Non Line Of Sight}, NLOS)
e calcolare di conseguenza la potenza del segnale in ricezione.
In~\cite{4020783}, invece, le informazioni sugli edifici erano codificate direttamente all'interno dell'ambiente sviluppato, chiamato
EGRESS (\textit{Enviroment for Generating REalistic Scenarios for Simulations}), e permettavano di definire tre tipologie di edifici e altrettanti
percorsi per simulare la mobilità dei nodi; il modello permetteva anche di configurare, con alcuni limiti, diversi tipi di materiale dei pavimenti e dei muri degli edifici.
Gli autori di~\cite{Carpenter:2015:OMI:2756509.2756512} estraevano le informazioni per la geometria degli edifici da una piattaforma gratuita
chiamata OpenStreetMap (OSM)~\cite{osmWebsite}; poi queste venivano elaborate da un software per la simuazione della mobilità di veicoli
che generava un file contenente i dati sugli edifici e un altro per la mobilità dei veicoli;
il modello di propagazione teneva conto della distanza interna percorsa del segnale e del numero di muri esterni che questo attraversava.
%
