%!TEX TS-program = xelatex
%!TEX encoding = UTF-8 Unicode
%!TeX spellcheck = it_IT
%!TEX root = ../tesi.tex

\chapter{Introduzione}\label{chap:introduction}
% \section{Contesto}\label{sec:contesto}
% scaletta:
% - introdurre le vanet
% - importanza della propagazione/protocolli broadcast
% - importanza delle simulazioni
% - importanza di simulazioni veritiere
% - (~) modelli per rendere migliori le simulazioni
% - importanza degli ostacoli
% - modellazione di ambiente 3d
% - struttura documenti tesi
I progressi tecnologici dell'ultimo decennio nell'hardware, nel software e nelle telecomunicazioni hanno permesso
la larga diffusione di unità computazionali all'interno degli autoveicoli. % , anche in quelli di fascia medio-bassa.
Questo ha portato a un incremento dell'interesse della ricerca scientifica in questo campo e, in particolare,
sullo studio delle reti veicolari o VANET (\textit{Vehicular Ad-hoc Networks}).

Le VANET rappresentano una famiglia delle reti Ad-hoc mobili (\textit{Mobile Ad-hoc Network}, MANET)
nelle quali i movimenti sono strutturati, i veicoli possono essere consapevoli della propria posizione nello spazio (ad esempio tramite geolocalizzazione GPS)
e spesso sono equipaggiati con attrezzatura che permette le comunicazioni inter-veicolari (es.~antenne radio).
Queste reti sono alla base di una vasta gamma di necessità, che spaziano dalla sicurezza stradale alle applicazioni multimediali,
dalla distribuzione dei dati sul traffico allo sviluppo di una rete infrastrutturale urbana.

Una caratteristica comune è la necessità di un sistema che propaghi le informazioni
con il minor ritardo possibile; per esempio un pericolo inatteso sulla careggiata potrebbe causare l'impossibilità di proseguire
lungo il percorso ed è necessario informare i veicoli che seguono.
Oppure all'interno di una trasmissione di contenuti multimediali (giochi mutiplayer, video) fra due o più veicoli
alte latenze potrebbero inficiare la fruizione dei contenuti~\cite{1580935}~\cite{PantelW02}.
Una delle soluzioni proposte si chiama Fast Broadcast: un protocollo per ridurre il tempo di propagazione
di un messaggio da una sorgente a una destinazione tramite una stima del raggio trasmissivo effettivo e
il conseguente utilizzo di questa per ridurre il numero di salti necessari al raggiungimento dell'obiettivo~\cite{Palazzi07howdo}.

La valutazione di protocolli di questo tipo su scenari urbani con migliaia di agenti risulta
problematica principalmente a causa dei costi proibitivi (oltre alle implicazioni sulla privacy)
che comporterebbe una sperimentazione nel mondo reale.
Spesso i ricercatori ricadono, quindi, sull'utilizzo di modelli simulati eventualmente integrati con risultati ottenuti
da esperimenti sul campo in ambienti ridotti e controllati.
In particolare, l'accuratezza dei modelli di propagazione e di mobilità rappresenta la chiave per una buona valutazione delle prestazioni
dei protocolli di rete veicolari~\cite{4020783}.
All'interno di scenari urbani e suburbani gli edifici ostruiscono la naturale propagazione di un segnale radio nello spazio
e, di conseguenza, al fine di eseguire simulazioni più affidabili questa ostruzione non può essere ignorata.
Una fra le diverse soluzioni proposte negli anni permette di calcolare l'attenuazione del segnale per un singolo edificio
in funzione del numero di intersezioni con le pareti esterne e la distanza interna percorsa~\cite{5720204}.
Qualche anno più tardi, il modello è stato ripreso, implementato per uno software di simulazioni
e integrato con informazioni reali sulla geometria degli edifici e sulla topologia stradale~\cite{Carpenter:2015:OMI:2756509.2756512}.
Un aspetto positivo di questo modello risiede nella sua implementazione, rilasciata
come modulo aggiuntivo per Network Simulator 3 (ns-3)~\cite{ns3Website},
ultima versione della nota famiglia di simulatori di rete, successore di ns-2
e l'unica a essere attualmente mantenuta.
ns-3 è assieme a \omnet~\cite{omnetWebsite} uno dei simulatori più diffusi
in ambito accademico e di ricerca, grazie alla loro natura \textit{open-source},
all'ambiente di configurazione delle simulazioni in \Cpp{} e a una comunità di sviluppo attiva.
Considerato questo e la presenza di una implementazione sia del modello a ostacoli
che del protocollo Fast Broadcast per ns-3 la scelta del simulatore da utilizzare
è ricaduta su quest'ultimo.

I principali contributi di questo lavoro si possono così riassumere.
\setlist{nolistsep}
\begin{itemize}[noitemsep]
	\item La valutazione del protocollo Fast Broadcast in un contesto urbano reale,
				utilizzando mappe geografiche e dati sugli ostacoli presenti di due città con caratteristiche topologiche diverse (Sezione ~\ref{sec:configurazione-la-pd}).
	\item La proposta e l'analisi di uno scenario nel quale alcuni veicoli non hanno la possibilità di partecipare allo scambio di messaggi
				e nel quale è stata prevista la presenza di sensori in aiuto alla propagazione (Sezione ~\ref{sec:configurazione-sensori}).
	\item Lo studio di un modello per l'ombreggiatura da ostacoli presente in letteratura e una sua possibile estensione nello spazio tridimensionale (Capitolo ~\ref{chap:modello-a-ostacoli}).
\end{itemize}
%
\paragraph{}
Il presente lavoro è strutturato come segue.
Nel capitolo successivo (Capitolo~\ref{chap:protocollo-fast-broadcast}) verrà presentato il protocollo Fast Broadcast e una proposta di estensione in due dimensioni.
Il Capitolo~\ref{chap:modello-a-ostacoli} analizzerà il modello a ostacoli partendo da una breve panoramica sulle soluzioni presenti in letteratura,
entrerà nel dettaglio del modello adottato per concludere presentando una possibile soluzione per rendere il modello tridimensionale.
Il Capitolo~\ref{chap:applicativi} darà una veloce panoramica sui software utilizzati e il metodo utilizzato per la creazione degli scenari.
Il Capitolo~\ref{chap:simulazioni} illusterà i diversi gruppi di simulazioni effettuati e i relativi risultati ottenuti.
Infine, il Capitolo~\ref{chap:conclusioni}
si concluderà riassumento i risultati ed evidenziando alcuni possibili sviluppi.
%
