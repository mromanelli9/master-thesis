%!TEX TS-program = xelatex
%!TEX encoding = UTF-8 Unicode
%!TeX spellcheck = it_IT
%!TEX root = ../tesi.tex

\chapter{Conclusioni}\label{chap:conclusioni}
Lo scopo di questo lavoro è stato valutare il protocollo Fast Broadcast per la propagazione dei messaggi
all'intero di un contesto urbano reale, con particolare attenzione all'impatto dell'effetto ombreggiatura derivato dalla
presenza di ostacoli.
Dalle prove effettuate è sorto evidente come questo sia un fattore chiave nelle simulazioni delle VANET,
specialmente se si considera uno scenario cittadino dove abitazioni, uffici, ospedali, edifici in genere sono sempre presenti.
L'utilizzo di dati reali sulla posizione e/o mobilità dei veicoli è altresì decisiva,
permettendo di cogliere effetti causati dalle caratteristiche intrinseche dell'ambiente sottostante.
Nel caso specifico, si è visto come a parità di portata del raggio trasmissivo la struttura
del centro cittadino di Padova influisce maggiormente sullo scambio di messaggi
rispetto allo scenario della città di Los Angeles,
pertanto per ottenere dei risulati uniformi, o se si vuole astrarre da uno scenario in particolare,
è necessario utilizzare una trasmissione con una portata maggiore.	% assunzioni forti
Il protocollo Fast Broadcast si comporta come previsto, confermando che una stima dinamica sul mezzo
tramissivo permette di coprire più veicoli all'interno dell'area prefissata e in minor tempo.
I risultati mostrano tuttavia che, in alcuni casi, l'effetto dell'ombreggiatura
potrebbe impattare in modo maggiore rispetto allo non stimare correttamente il mezzo fisico. % --> boh
È opinione degli autori che un adeguato modello di rappresentazione dell'ambiente sia
necessario durante lo studio di questa tipologia di scenari in quanto la presenza degli ostacoli non può essere esclusa.
L'utilizzo di dati effettivi sulla geometria di questi permette di ridurre ulteriormente il divario fra simulazioni software e prove sul campo.
In aggiunta, si è visto come la bassa copertura dei veicoli può essere migliorata in modo considerevole tramite l'aggiunta
di dispositivi a basso costo, come per esempio vari tipi di sensori, incrementando però anche il traffico dati che si genera.

Sebbene questi scenari prendevano in considerazione la topologia stradale nel posizionamento dei veicoli,
permettere a questi di muoversi garantirebbbe risultati più affidabili;
questo potrebbe essere fatto seguendo uno dei diversi modelli di mobilità proposti in letteratura,
oppure utilizzando tracce reali rese disponbili dagli utenti su alcune piattaforme di navigazione.
Per quanto riguarda la valutazione del protocollo, sarebbe interessante studiarne
le prestazioni in presenza di diversi messaggi di inoltro generati da veicoli diversi,
come anche in presenza di un servizio a livello applicativo che produca un traffico dati costante
che il protocollo debba gestire.

In relazione al modello a ostacoli, la modifica tridimensionale proposta è una delle
possibili soluzioni e ulteriori prove sono sicuramente necessarie per determinare
in che misura l'utilizzo della terza dimensione impatti sulle simulazioni.
Inoltre, sarebbe utile integrare nel simulatore ns-3 in maniera più profonda le informazioni sugli ostacoli
in modo che non siano utilizzabili anche da altri moduli,
ad esempio per permettere ai nodi di essere consapevoli di eventuali ostruzioni durante il movimento.
(si pensi a droni che si muovono intorno a degli edifici).
% sviluppo futuri
% - mobilità con traccie reali (maps ?)
% - esperimenti veri
% - sviluppo del mod. 3d
% - qualche sviluppo sullo scenario sensori
