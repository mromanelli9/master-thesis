%!TEX TS-program = xelatex
%!TEX encoding = UTF-8 Unicode
%!TeX spellcheck = it_IT
%!TEX root = ../tesi.tex

\chapter{Conclusioni}\label{chap:conclusioni}
Lo scopo di questo lavoro è stato valutare l'impatto dell'effetto ombreggiatura derivato dalla
presenza di edifici sulla propagazione di messaggi in una rete VANET all'intero di uno scenario urbano;
oltre a questo, osservare il comportamento del protocollo Fast Broadcast sotto queste condizioni,
un protocollo che stima dinamicamente il raggio trasmissivo per propagare il messaggio
con meno salti possibili.
Dalle prove effettuate è sorto evidente come l'effetto ombreggiatura causato dagli ostacoli
sia un fattore chiave nelle simulazioni delle VANET, specialmente se si è in uno scenario
urbano o suburbano dove abitazioni, uffici, ospedali, edifici in genere sono sempre presenti.
L'utilizzo di dati reali sulla posizione e/o mobilità dei veicoli è altresì decisiva,
permettendo di cogliere effetti causati dalle caratteristiche intrinseche dell'ambiente sottostante.
% TODO controllare dopo i risultati totali
Nel caso specifico, si è visto come a parità di portata del raggio trasmissivo la struttura
del centro cittadino di Padova influisce maggiormente sullo scambio di messaggi
rispetto allo scenario della città di Los Angeles;
pertanto per ottenere dei risulati uniformi, o se si vuole astrarre da uno scenario in particolare,
è necessario utilizzare una trasmissione con una portata maggiore.	% assunzioni forti
Il protocollo Fast Broadcast si comporta come previsto, confermando che una stima dinamica sul mezzo
tramissivo permette di coprire più veicoli all'interno dell'area prefissata e in minor tempo.
I risultati mostrano, tuttavia, che, in alcuni casi, l'effetto dell'ombreggiatura
potrebbe impattare in modo maggiore rispetto allo non stimare correttamente il mezzo fisico. % --> boh
È opinione degli autori che un adeguato modello di rappresentazione dell'ambiente sia
sempre necessario durante lo sviluppo di un protocollo di questo tipo
e che la presenza degli ostacoli non può essere esclusa.
% TODO inserire risultati scenario sensori

Sebbene questi scenari prendevano in considerazione la topologia stradale nel posizionamento dei veicoli,
sarebbe necessario che i veicoli avessero la capacità di muoversi, seguendo uno dei diversi modelli di mobilità
proposti in letteratura;
ancora meglio sarebbe la possibilità di utilizzare tracce reali in modo da cogliere eventuali problematiche
che potrebbero non presentarsi utilizzando un modello teorico.
Un'altra possibile continuazione di questo lavoro consiste nell'effettuare degli esperimenti reali
degli scenari proposti, anche se su aree ridotte, con lo scopo di determinare l'effettiva efficacia e
validità del modello a ostacoli.
Sempre in relazione a quest'ultimo, la modifica tridimensionale proposta è una delle
possibili soluzioni e ulteriori prove sono sicuramente necessarie per determinare
se e quanto l'utilizzo della terza dimensione impatti sui risulati delle simulazioni.
% sviluppo futuri
% - mobilità con traccie reali (maps ?)
% - esperimenti veri
% - sviluppo del mod. 3d
% - qualche sviluppo sullo scenario sensori
