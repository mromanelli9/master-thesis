%!TEX TS-program = pdflatex
%!TEX encoding = UTF-8 Unicode
%!TeX spellcheck = it_IT
%!TEX root = ../tesi.tex
%
\chapter{Simulazioni}
% INTRO: cosa c'è in questa sezione?
Dopo aver illustrato il protocollo in esame e il modello di propagazione utilizzato, si passa ora alla fase di valutazione.
Verranno prima illustrati brevemente il programma utilizzato per le simulazioni e gli applicativi di supporto.
Il capitolo procederà, quindi, nel dettaglio delle varie simulazioni effettuate analizzando i relativi risultati.
%
\section{Applicativi}
%
% \section{Panoramica}
% - come sono strutturate le simulazioni?
% - è importante fare simulazioni differenti (barichello e scenario reale, vedi paper)?
\section{Configurazione a griglia}
% Simulazioni codice barichello:
%  - struttura simulazioni
%  - grafici
%  - risulati
Il primo gruppo di simulazioni ha lo scopo di analizzare il comportamento del protocollo Fast Broadcast con e senza edifici utilizzando uno scenario conosciuto;
per questo motivo, la configurazione è la stessa presentata nella tesi originale \cite{Barichello2017propagazione}.
Lo scenario è una città fittizia con strade a griglia in stile Manhattan di lunghezza 4000 metri e distanti l'una dall'altra di 300 metri.
I veicoli sono disposti a 12 metri di distanza per un totale di 8064.
Seguendo l'idea originale, è stata definita anche un circonferenza di ragggio pari a 1000$\pm$12 metri, utilizzata per definire alcune metriche che verranno descritte in seguito.
Il raggio trasmissivo effettivo assume due valori possibili: 300 metri o 500 metri.
Il tutto è riassunto in Tabella?.

Per valutare il protocollo sono presi in esame diversi parametri: la copertura totale e la copertura sulla circonferenza (in percentuale),
il numero di salti necessari per raggiungere il bordo della griglia, il numero totale di messaggi di inoltro ricevuti e inviati.
In modo da aver un metro di paragone per il protocollo Fast Broadcast, quest'ultimo è stato comparato con altri due metodi, chiamati STATIC300 e STATIC500,
che utilizzano una stima fissa del raggio trasmissimo, di 300 e 500 metri rispettivamente.

Infine, oigni scenario è stato eseguito per 50 volte. % così non mi piace


% TODO tabella riassuntiva configurazione scenario
