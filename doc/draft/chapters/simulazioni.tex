%!TEX TS-program = pdflatex
%!TEX encoding = UTF-8 Unicode
%!TeX spellcheck = it_IT
%!TEX root = ../tesi.tex
%
\chapter{Simulazioni}
% INTRO: cosa c'è in questa sezione?
Dopo aver illustrato il protocollo in esame e il modello di propagazione utilizzato, si passa ora alla fase di valutazione.
Verranno prima illustrati brevemente il programma utilizzato per le simulazioni e gli applicativi di supporto.
Il capitolo procederà, quindi, nel dettaglio delle varie simulazioni effettuate analizzando i relativi risultati.
%
\section{Applicativi}
%
% \section{Panoramica}
% - come sono strutturate le simulazioni?
% - è importante fare simulazioni differenti (barichello e scenario reale, vedi paper)?
\section{Configurazione a griglia}
\subsection{Panoramica}
% Simulazioni codice barichello:
%  - struttura simulazioni
%  - grafici
%  - risulati
Il primo gruppo di simulazioni ha lo scopo di analizzare il comportamento del protocollo Fast Broadcast con e senza edifici utilizzando uno scenario conosciuto;
per questo motivo, la configurazione è la stessa presentata nella tesi originale \cite{Barichello2017propagazione}.
Lo scenario è una città fittizia con strade a griglia in stile Manhattan di lunghezza 4000 metri e distanti l'una dall'altra di 300 metri.
I veicoli sono disposti a 12 metri di distanza per un totale di 8064.
Seguendo l'idea originale, è stata definita anche un circonferenza di ragggio pari a 1000$\pm$12 metri, utilizzata per definire alcune metriche che verranno descritte in seguito.
Il raggio trasmissivo effettivo assume due valori possibili: 300 metri o 500 metri.
Il tutto è riassunto in Tabella~\ref{tab:parametri-simulazioni-barichello}.

Per valutare il protocollo sono presi in esame diversi parametri: la copertura totale e la copertura sulla circonferenza,
il numero di salti necessari per raggiungere il bordo della griglia, il numero totale di messaggi di inoltro ricevuti e inviati.
In modo da aver un metro di paragone per il protocollo Fast Broadcast, quest'ultimo è stato comparato con altri due metodi, chiamati STATIC300 e STATIC500,
che utilizzano una stima fissa del raggio trasmissimo, di 300 e 500 metri rispettivamente.

Per quanto riguarda gli edifici, il file dati contenente i poligoni è stato appositamente creato, in modo da avere un edificio di forma quadrata di lato 300 metri
posizionato all'interno di ogni ``blocco'' della griglia, come mostrato in Figura~\ref{fig:griglia-dettaglio}.
Gli edifici sono distanziati di 5 metri dalla strada su ogni lato, in modo da non far aderire i muri degli edifici al bordo la strada.

A fini statistici, la simulazione di ogni scenario è stata ripetuta per 50 volte. % così non mi piace
\begin{figure}[!h]
	\centering
	\begin{center}
		\includegraphics[width=.4\textwidth]{griglia-dettaglio.png}	% l'immagine fa un po' schifo --> già meglio
		\hspace{8pt}
		\includegraphics[width=.345\textwidth]{griglia-dettaglio-2.png}
	\end{center}
	\label{fig:griglia-dettaglio}\caption{Dettaglio della configurazione a griglia: in nero le strade e in blu gli edifici.}
\end{figure}
%
% TODO: verificare stile tabella: così mi piace?
\begin{table}[!h]
	\begin{center}
	  \begin{tabular}{ | m{.5\linewidth} | r  c | }
			\hline
			Parametro											&			\multicolumn{2}{c|}{Valore}			\\ \hline \hline
			Configurazione								&			\multicolumn{2}{c|}{A griglia}		\\ \hline
			Lunghezza delle strade				&			4000 								& metri		\\ \hline
			Distanza fra le strade				&			300 								& metri		\\ \hline
			Distanza fra i veicoloi 			&			12 									& metri		\\ \hline
			Raggio trasmissivo effettivo	&			300 - 500						& metri		\\
			\hline
	  \end{tabular}
	\end{center}
	\caption{Configurazione dei parametri per le simulazioni}
	\label{tab:parametri-simulazioni-barichello}
\end{table}
%
\subsection{Risulati}

















%
