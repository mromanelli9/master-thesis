%!TEX TS-program = pdflatex
%!TEX encoding = UTF-8 Unicode
%!TeX spellcheck = it_IT
%!TEX root = ../tesi.tex

\chapter{Protocollo Fast Broadcast}
La tecnica del Fast Broadcast~\cite{Palazzi07howdo} si compone di due fasi. La prima, chiamata ``Fase di stima'', permette a ogni veicolo di avere una stima aggiornata
del proprio raggio trasmissivo. La seconda, invece, è detta ``Fase di inoltro'' viene attivata nel momento in cui si rende necessario inviare un messaggio
al resto dei veicoli presenti nell'area di interesse.

\section{Fase di stima}
Durante questa fase, ogni veicolo effettua una stima del suo raggio trasmissivo, sia di fronte che alle sue spalle, attraverso l'uso di messaggi \textit{Hello}.
Per ottenere una stima sempre aggiornata, il tempo è suddiviso in turni e le informazioni raccolte durante un certo turno
rimangono attive per tutto il turno successivo, per poi essere scartate.
Una brave durata dei turni permette di cogliere meglio variazioni del raggio trasmissivo mantendendo tuttavia un elevato scambio di messaggi;
gli autori suggeriscono un tempo pari a 1 secondo.

Le informazioni sull stima sono rappresentate dai campi \textit{Current-turn Maximum Front Range} (CMFR) e \textit{Current-turn Maximum Back Range} (CMBR).
Il primo esprime la stima della massima distanza in avanti dalla quale un altro veicolo nell'area di interesse può ricevere messaggi che provengono dal veicolo considerato.
Il secondo, invece, stima la massima distanza all'indietro.
Questi valori sono costantemente aggiornati con i valori ricevuti nei messaggi Hello fino alla fine del turno;
dopodiché vengono salvati nei campi \textit{Latest-turn Maximum Front Range} (LMFR) e \textit{Latest-turn Maximum Back Range} (LMBR) rispettivamente.
Questo permette di combinare una stima calcolata nel tempo (con diversi messaggi Hello) con le ultime informazioni sul mezzo trasmissivo.

Nel dettaglio, la procedura di invio di un messagio Hello prevede inizialmente che veicolo aspetti un tempo d'attesa casuale e,
dopo aver verificato l'assenza di altre trasmissioni in corso e/o collisioni, proceda poi all'invio del messaggio contente
la stima del raggio trasmissivo in avanti.
Altre informazioni utili al protocollo vengono inserite all'iterno, come ad esempio la posizione aggiornata del veicolo.
