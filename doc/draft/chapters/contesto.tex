%!TEX TS-program = pdflatex
%!TEX encoding = UTF-8 Unicode
%!TeX spellcheck = it_IT
%!TEX root = ../tesi.tex
%
\chapter{Contesto e lavori correlati}
%
\section{Modelli di radiopropagazione}
Un modello di radiopropagazione (MRP) simula gli effetti dell'attenuazione del segnale radio (segnali elettromagnetici nello spazio libero o etere, in contrasto con la propagazione guidata)
dovuta alla distanza, cammini multipli per effetto della riflessione, ombreggiatura causata da grandi ostacoli come edifici.
L'utilizzo di un'idonea rappresentazione per questo tipo di ostacoli è, quindi, necessaria nel contesto di simulazioni di reti VANET (\textit{Vehicular Ad-hoc Networks}) in ambienti urbani
e suburbani.

Nel corso degli anni, diversi MRP sono stati proposti.
Il più semplice di questi si chiama modello a disco unitario (\textit{unit-disk model}), nel quale i veicoli possono comunicare fra loro se si trovano entro una certa soglia
di distanza, mentre non possono altrimenti~\cite{6554832}.
Un modello molto utilizzato nelle simuazioni di VANET è il modello a doppio raggio (\textit{Two-ray ground-reflection model}),
nel quale il segnale in fase di ricezione è composto da una componente in linea diretta e una seconda derivante dalla riflessione causata dal terreno~\cite{DBLP:books/daglib/0091821}.
In~\cite{Schmitz:2006:ERW:1164717.1164730} e in~\cite{Souley2005RealisticUS}, modelli più sviluppati prendono in considerazione anche le properità riflettenti delle superfici e degli ostacoli.

Tuttavia, un approcio diretto di questo tipo difficilmente scala al numero di nodi necessario in un classico scenario VANET e per questo motivo
spesso questi modelli si affidano a una fase di pre-elaborazione che può richiedere tempi elevati~\cite{Stepanov:2008:IMR:1293378.1293656}.
Per ovviare a questo problema, alcune proposte astraggono dalle informazioni sui singoli edifici, modellando l'ambiente urbano in modo omogeneo
così da creare un modello analitico per l'ombreggiatura~\cite{1492678}.
Questo tipo di modelli riducono il costo computazionale e generalmente si raffrontano bene con i risultati reali;
ciò nonostante, non riescono a catturare effetti a livello mesoscopico, come eventuali spazi fra edifici che permetterebbero trasmissioni a corto raggio~\cite{Giordano:2010:CST:1860058.1860065}.

Sono affetti dallo stesso problema anche  i modelli (puramente) stocastici; questi modellano l'effetto dell'ombraggiatura tramite distribuzione stocastiche,
fra cui Rice, Rayleigh, Nakagami-m, lognormale Weibull~\cite{6554832}~\cite{Rappaport:2001:WCP:559977}, in quanto non tengono in considerazione la geometria urbana sottostante.
%
\section{Modellazione di ostacoli}







%
