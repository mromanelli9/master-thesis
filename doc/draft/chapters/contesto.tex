%!TEX TS-program = pdflatex
%!TEX encoding = UTF-8 Unicode
%!TeX spellcheck = it_IT
%!TEX root = ../tesi.tex
%
\chapter{Contesto e lavori correlati}
%
\section{Modelli di radiopropagazione}
Un modello di radiopropagazione (MRP) simula gli effetti dell'attenuazione del segnale radio (segnali elettromagnetici nello spazio libero o etere, in contrasto con la propagazione guidata)
dovuta alla distanza, cammini multipli per effetto della riflessione, ombreggiatura causata da grandi ostacoli come edifici.
L'utilizzo di un'idonea rappresentazione per questo tipo di ostacoli è, quindi, necessaria nel contesto di simulazioni di reti VANET (\textit{Vehicular Ad-hoc Networks}) in ambienti urbani
e suburbani.

Nel corso degli anni, diversi MRP msono stati proposti.
Il più semplice di questi si chiama modello a disco unitario (\textit{unit-disk model}), nel quale i veicoli possono comunicare fra loro se si trovano entro una certa soglia
di distanza, mentre non possono altrimenti~\cite{6554832}.
Un modello molto utilizzato nelle simuazioni di VANET è il modello a doppio raggio (\textit{Two-ray ground-reflection model}),
nel quale il segnale in fase di ricezione è composto da una componente in linea diretta e una seconda derivante dalla riflessione causata dal terreno~\cite{DBLP:books/daglib/0091821}.

\section{Modellazione di ostacoli}







%
