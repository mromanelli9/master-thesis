%!TEX TS-program = pdflatex
%!TEX encoding = UTF-8 Unicode
%!TeX spellcheck = it_IT
%!TEX root = ../tesi.tex
%
\chapter{Software}
%
\section{Network Simulator 3}
Network Simulator 3 (ns-3)~\cite{ns3Website} è un simulatore a eventi discreti utilizzato principalmente in ambito accademico e di ricerca,
scritto in \Cpp e Python; è un software libero distribuito sotto licenza GNU GPLv2.
Creato nel 2006 da un gruppo coordinato da Tom Henderson, George Riley, Sally Floyd e Sumit Roy con lo scopo
di ovviare ad alcune grosse limitazioni della precedente versione (ns-2), come ad esempio la necessità di utilizzare
un linguaggio di \textit{scripting} dedicato per modellare le simulazioni o una maggiore scalabilità~\cite{Henderson:2006:NPG:1190455.1190468}.
Attualmente è l'unica versione della serie a essere sviluppata.
È d'obbligo menzionare che, sebbene tutte le versione siano state scritte in \Cpp, questa non è retrocompatibile con le precedenti.
%
\subsection{Concetti chiave}
Alcuni concetti chiave:
\begin{itemize}
	\item In ns-3, l'unità elementare è chiamata \textit{nodo}: questa fornisce i metodi per gestire i dispositivi nelle simulazioni.
	\item Sui nodi girano delle \textit{applicazioni}: queste eseguono delle azioni per giungere a un obbiettivo.
				Per esempio, le applicazioni \textsf{UdpEchoClientApplication} e \textsf{UdpEchoServerApplication} compongono
				un'aplicazione client/server per gestire e scambiare pacchetti nella rete.
	\item Le comunicazioni fra i nodi avvengono tramite degli speciali \textit{canali};
				i canali modellano diversi concetti, dai più semplici come un collegamento su cavo a quelli più complessi
				come uno switch Ethernet.
				Il canale \textsf{CsmaChannel}, ad esempio, modella un mezzo tramissivo con accesso CSMA.
	\item Sui nodi si installano dei \textit{dispositivi di rete} che permettono la comunicazione fra questi attraverso i canali.
				Un nodo può essere connesso a più canali attraverso diversi dispositivi di rete.
				Per utilizzare il canale \textsf{CsmaChannel} è presente il dispositivo \textsf{CsmaNetDevice}.
\end{itemize}
